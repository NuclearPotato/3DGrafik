\section{Report}

% NAMES AND STUDENT NRS GO HERE GUYS
Sune Døssing - 20115515


% How did you represent and draw the blocks, block-outline and player? why?

Our blocks are represented as a block object, which holds a block type and 4 vertices.
The initial amount of blocks is coded as an X and Y amount in the javascript code,
and the layout of these is made in proportion.
Blocks are drawn as two triangles using triangle fan, creating triangle of v1,v2,v3 and v2,v3,v4.
Block-outline and player vertices are hardcoded at the end of the buffer, after the blocks.

We would have used one buffer for blocks, 
one for outline and one for player character, 
but could not get multiple buffers to work.

% How was your block-material/color implemented? why?

Each blocktype has a predefined color used when rendering each block.
Color gradient is calculated in the fragment shader, 
dependant on the distance of the fragment (fPosition) to the center of its parent block,
which is calculated in the vertex shader and passed on to the fragment shader.

% Which attributes, uniform and varying did you use, and why?

Variables given to the shaders by javavscript code are uniform, 
as these are not meant to be changed during rendering.
Varying variables can be changed during GPU rendering and are 
used to pass information between the vertex shader and fragment shader.

% How much buffer-data needs to be updated interactively and when? 
% Did you implement the most optimal approach?

% What does your vertex shaders do?

% What does your fragment shader do?
The fragment shader decides the color of individual pixels, 
and uses the centerpos to create a circular gradient.


%Please indicate which browser your solution runs on
Our solution runs on Firefox (and Nightly), Internet Explorer (and Edge), and Chrome.
Chrome runs very slowly, while the others do not.