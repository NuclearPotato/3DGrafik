\section{Report}

% NAMES AND STUDENT NRS GO HERE GUYS
Rasmus Mosbech Jensen - 20105109
Sune Døssing - 20115515


% How did you represent and draw the blocks, block-outline and player? why?

Our blocks are represented as a block object, which holds a block type and 4 vertices.
The initial amount of blocks is coded as an X and Y amount in the javascript code,
and the layout of these is made in proportion.
Blocks are drawn as two triangles using triangle fan, creating triangle of v1,v2,v3 and v2,v3,v4.
Block-outline and stickman are hardcoded at the end of the buffer, after the blocks.

In our rendering function, we first iterate over each of our blocks, calling drawArrays individually for each of them. The mouse-over box is then highlighted, and then stickman.

% How was your block-material/color implemented? why?

Each blocktype has a predefined color used when rendering each block.
Color gradient is calculated in the fragment shader, 
dependant on the distance of the fragment (fPosition) to the center of its parent block,
which is calculated in the vertex shader and passed on to the fragment shader.

% Which attributes, uniform and varying did you use, and why?


Our stickman uses a uniform vec4 (actually only vec2) which is added to each of the stickmans vertex positions to allow movement.
This allows us to only put vertex positons into the buffer once and update it only with an offset. 
We also use a wavedistance variable which is incremented by the javascript code and determines how far the wave effect has gotten.

We use a uniform variable to determines which blocks are affected by the gradient effect, and by offset.
And we use uniform variables to get corners used to calculate a blocks middle position, and center of a mouse click within the vertex shader.

Necessary data from the vertex shader, 
such as wave distance (for wave effect) and the blocks middle position (for gradient), 
are shared with the fragment shader via varying variables.

% How much buffer-data needs to be updated interactively and when? 

The buffer has enough space for every blocks 4 vertices, the 4 vertices of the mouse-over box, and 8 vertices for the stickman figure.
Buffer is initially filled with the blocks using the handleBuffer() function, 
that iterates through our array of blocks and adds the corners of each block, one block at a time.

In the render function, if a mouseclick has happened, 
the clicked block will be recolored (by changing the color buffer) to match the new type and has its object values updated to reflect this.

The highlighting box is put in the buffer every time the render function is called.

The stickman is placed into the buffer once and never again. Its vertices are never updated in the buffer, but are instead moved by adding an offset to them in the vertex shader.

% Did you implement the most optimal approach?

Air blocks are treated the same way as all other blocks, 
but could actually have been not-rendered to avoid having to do vertex and fragment shading for them, and just leave the blank canvas instead.

Our highlighting box is submitted to the buffer every time the render function is called. 
This is much slower than submitting it once initially and moving it around using a uniform vector offset.

The middle positon for each block is calculated in every vertex, but only changes with every call to drawArrays.

Our solution uses only one buffer, one vertex shader, and one fragment shader. 
Many conditionals could have been averted if we used multiple. 
We would not need to check in the vertex shader if we are dealing with stickman or a block, if we had a dedicated vertex shader for the stickman.
This also means many uniform variables are handled even though they are not needed for every vertex.

% What does your vertex shaders do?

The vertex shader checks if it is handling stickman, and adds the offset to its position.
It calculates the middle point of the vertexs parent block, and gives this to the fragment shader.
It also handles wave effect, by getting the middle point between mouseclick and current vertex, and changing its position accordingly.

% What does your fragment shader do?
The frament shader calculates distance of each fragment to the center of parent block (given as centerpos by vertex shader), and changes color accordingly, to create gradient.
It also changes color depending on how strong the wave effect is, which is also given by the vertex shader.

%Please indicate which browser your solution runs on
Our solution runs on Firefox (and Nightly), Internet Explorer (and Edge), and Chrome.
Chrome runs very slowly, while the others do not. Edge generates an "unspecified error", which doesnt seem to cause any problems.